% Created 2024-10-16 śro 21:35
% Intended LaTeX compiler: pdflatex
\documentclass[../../main.tex]{subfiles}

% \usepackage[a4paper, margin=3cm]{geometry}
% \usepackage{amssymb} // not working

\usepackage[T1]{fontenc}
\usepackage[utf8]{inputenc}
\usepackage{graphicx}
\usepackage{longtable}
\usepackage{wrapfig}
\usepackage{rotating}
\usepackage[normalem]{ulem}
\usepackage{amsmath}
\usepackage{capt-of}
\usepackage{hyperref}
\usepackage{siunitx}
\usepackage{float}
\usepackage[polish]{babel}

\graphicspath{{../}}
\author{Wojciech Paderewski}
\date{\today}
\title{Koncepcja układu}
\hypersetup{
 pdfauthor={Wojciech Paderewski},
 pdftitle={Koncepcja układu},
 pdfkeywords={},
 pdfsubject={},
 pdflang={Polish}}
 
\begin{document}


\subsubsection{Dobór złącza}
Złącze USB musi posiadać przynajmniej 12 wyprowadzeń, ponieważ dopiero w takim układzie na złączu są linie D+ i D-, czyli linie danych.
Wybrano złącze z 16 wyprowadzeniami, ponieważ takie było dostępne w sklepie.
\subsubsection{Opis podłączenia}
By ustawić napięcie komunikacji USB-C na \SI{3.3}{\volt}, zastosowano rezystory podciągające R1 i R2 o wartości \SI{5.1}{\kilo\ohm}.
Do podłączenia wykorzystano parę różnicową, by połączyć linie D+ i D- z ESP32-S3, w celu zminimalizowania zakłóceń CMN (Common Mode Noise).
\subsubsection{Zabezpieczenia ESD}
W celu zabezpieczenia linii przed przepięciami, zastosowano transil PUSB3AB4Z firmy Nexperia. Diody te mają wystarczająco duże opakowanie, by dało się je przylutować ręcznie. Napięciem roboczym jest \SI{3.3}{\volt},
a napięcie stabilizacji wynosi \SI{5}{\volt}.

Mimo że jest to napięcie wyższe niż napięcie zasilania ESP32-S3, to nie powinno to stanowić problemu, 
ponieważ napięcie to pojawi się na krótki czas, a sam mikrokontroler ma również wbudowane zabezpieczenia przed przepięciami.

Wewnętrzne zabezpieczenia według karty katalogowej ESP32-S3:
\begin{itemize}
    \item Test Standard JS-001; HBM (Human Body Mode) ± 2000 V
    \item Test Standard JS-002; CDM (Charged Device Model) ± 1000 V
\end{itemize}
Wynika z tego, że złącze USB-C jest dość dobrze zabezpieczone przed przepięciami.
Schemat elektryczny złącza USB-C przedstawiono na rysunku \ref{fig:usb-c_schemat}.
\begin{figure}[H]
    \centering
    \includegraphics[width=0.7\textwidth]{usb-c_schemat.png}
    \caption{Schemat elektryczny złącza USB-C}
    \label{fig:usb-c_schemat}
\end{figure}
\end{document}
