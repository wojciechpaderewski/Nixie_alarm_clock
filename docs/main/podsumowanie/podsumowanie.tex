% Created 2024-10-16 śro 21:35
% Intended LaTeX compiler: pdflatex
\documentclass[../main.tex]{subfiles}

% \usepackage[a4paper, margin=3cm]{geometry}
% \usepackage{amssymb} // not working

\usepackage[T1]{fontenc}
\usepackage[utf8]{inputenc}
\usepackage{graphicx}
\usepackage{longtable}
\usepackage{wrapfig}
\usepackage{rotating}
\usepackage[normalem]{ulem}
\usepackage{amsmath}
\usepackage{capt-of}
\usepackage{hyperref}
\usepackage{siunitx}
\usepackage{float}
\usepackage[polish]{babel}

\graphicspath{{../}}
\author{Wojciech Paderewski}
\date{\today}
\title{Koncepcja ukladu}
\hypersetup{
 pdfauthor={Wojciech Paderewski},
 pdftitle={Koncepcja ukladu},
 pdfkeywords={},
 pdfsubject={},
 pdflang={Polish}}

\begin{document}

Celem niniejszej pracy inżynierskiej było opracowanie i wykonanie budzika synchronizowanego przez Wi-Fi z wykorzystaniem lamp Nixie.

W pracy omówiono zasadę działania lamp Nixie, mikrokontrolera ESP32-S3 oraz różnych serwerów czasu.
Wyjaśniono jakie są wady i zalety lamp Nixie, jakie są ich zastosowania. Przeanalizowano różne metody sterowania wyświetlaczami Nixie, w celu wybrania najlepszej dla projektu.
Wyjaśnione możliwości mikrokontrolera ESP32-S3 w zastosowaniach IoT oraz jego środowisko programistyczne.
Przedstawiono różne serwery czasu, ich zasady działania, wady i zalety. Szczegółowo opisano protokół NTP, który jest wykorzystywany w projekcie.

Na podstawie karty projektu i analizy wymagań oraz wykonania poszukiwań, przedstawiono założenia projektowe. W oparciu o nie powstała ogólna koncepcja układu,
a z niej powstał szczegółowy schemat wykonania projektu oraz jego podział na moduły. Następnie na podstawie testu zakupionych lamp Nixie, określone konkretne parametry które muszą spełniać wybrane elementy.
Bazując na tym podziale, przystąpiono wyboru elementów, które będą wykorzystane w projekcie. Następnie stworzono schematy elektryczne poszczególnych modułów i opisano sposób ich połączenia, z uwzględnieniem niezbędnych obliczeń.

W kolejnym etapie przystąpiono do implementacji projektu. Zaprojektowano i wykonano płytkę PCB z uwzględnieniem aspektów wizualnych oraz funkcjonalnych. Następnie zmontowano układ i przystąpiono do testów.
W trakcie testów napotkano problem z przetwornicą wysokiego napięcia, która nie osiagneła zakładanego przedziału regulacji i miała mniejszą sprawność niż założono, co powodowało nagrzewanie się przy górnej granicy zakresu. Problem został rozwiązany poprzez zwężenie przedziału regulacji napięcia wyjściowego.
Pozostałe testy przebiegły pomyślnie, a układ działał zgodnie z założeniami.

Dalsze prace nad opracowanym urządzeniem powinny obejmować przede wszystkim przeprojektowanie układu przetwornicy wysokiego napięcia, w celu uniknięcia problemów z nagrzewaniem się i poszerzenia zakresu regulacji. Można również dodać dodatkowe funkcje interface'u użytkownika, 
takie jak obsługa paska LED, regulacja napięcia wyjściowego z poziomu aplikacji, dodanie możliwości ustawienia alarmu w trybie offline. Można również wykonać obudowę dla układu,
aby zabezpieczyć go przed uszkodzeniami mechanicznymi oraz kurzem.

\end{document}
