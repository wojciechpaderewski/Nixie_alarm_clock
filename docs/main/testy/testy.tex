% Created 2024-10-16 śro 21:35
% Intended LaTeX compiler: pdflatex
\documentclass[../main.tex]{subfiles}

% \usepackage[a4paper, margin=3cm]{geometry}
% \usepackage{amssymb} // not working

\usepackage[T1]{fontenc}
\usepackage[utf8]{inputenc}
\usepackage{graphicx}
\usepackage{longtable}
\usepackage{wrapfig}
\usepackage{rotating}
\usepackage[normalem]{ulem}
\usepackage{amsmath}
\usepackage{capt-of}
\usepackage{hyperref}
\usepackage{siunitx}
\usepackage{float}
\usepackage[polish]{babel}

\graphicspath{{../}}
\author{Wojciech Paderewski}
\date{\today}
\title{testy}
\hypersetup{
 pdfauthor={Wojciech Paderewski},
 pdftitle={testy},
 pdfkeywords={},
 pdfsubject={},
 pdflang={Polish}}

\begin{document}
W tym rozdziale opisano testowanie poszczególnych 
modułów oraz całego układu, które pozwoliło na sprawdzenie poprawności działania oraz zidentyfikowanie błędów.

\subsection{Testy Przetwornicy HV}
Przetwornica HV jako najtrudniejszy moduł układu, została odizolowana na etap pierwszego uruchomiania rezystorem 0\si{\ohm}
w celu zminimalizowania ryzyka uszkodzenia lamp Nixie. Po podłączeniu zasilania przetwornica działa z napięciem wyjściowym 
ok 160V, co jest wartością oczekiwaną. Napięcie to jest stabilne, nie obserwuje się żadnych skoków napięcia.

% \begin{figure}[H]
%     \centering
%     \includegraphics[scale=0.3]{testy_przetwornicy_HV.png}
%     \caption{Testy przetwornicy HV}
% \end{figure}

Po napisaniu testowego programu do komunikacji z potencjometrem, przetestowano działanie regulacji. Regulacja działa poprawnie, faktyczny regulacja 
zakres regulacji wynosi ??? i ???;

% \begin{figure}[H]
%     \centering
%     \includegraphics[scale=0.3]{testy_przetwornicy_HV.png}
%     \caption{Testy przetwornicy HV}
% \end{figure}

% \begin{figure}[H]
%     \centering
%     \includegraphics[scale=0.3]{testy_przetwornicy_HV.png}
%     \caption{Testy przetwornicy HV}
% \end{figure}

\subsection{Testy układu sterowania}
Po pełnym złożeniu układu napisano prosty program który zapala i gasi wszystkie cyfry po kolei by przetestować działanie sterowanie
lampami nixie.

% \begin{figure}[H]
%     \centering
%     \includegraphics[scale=0.3]{testy_przetwornicy_HV.png}
%     \caption{Testy przetwornicy HV}
% \end{figure}

\subsection{Testy enkodera i buzzera}
Najpierw napisano prosty program do obsługi enkodera i przetestowano jego działanie. Działa wykrywanie obrotu oraz przycisk. Analogicznie
wgrano przykładowy program do odtwarzania dźwięków i przetestowano działania buzzera. Udało się zagrać melodie, ale buzzer okazał się cichszy
niż zakładano, ale nie powinno to stanowić problemu.

\end{document}


