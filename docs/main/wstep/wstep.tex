% Created 2024-10-16 śro 21:35
% Intended LaTeX compiler: pdflatex
\documentclass[../main.tex]{subfiles}

% \usepackage[a4paper, margin=3cm]{geometry}
% \usepackage{amssymb} // not working

\usepackage[T1]{fontenc}
\usepackage[utf8]{inputenc}
\usepackage{graphicx}
\usepackage{longtable}
\usepackage{wrapfig}
\usepackage{rotating}
\usepackage[normalem]{ulem}
\usepackage{amsmath}
\usepackage{capt-of}
\usepackage{hyperref}
\usepackage{siunitx}
\usepackage{float}
\usepackage[polish]{babel}

\graphicspath{{../}}
\author{Wojciech Paderewski}
\date{\today}
\title{Koncepcja ukladu}
\hypersetup{
 pdfauthor={Wojciech Paderewski},
 pdftitle={Koncepcja ukladu},
 pdfkeywords={},
 pdfsubject={},
 pdflang={Polish}}

 \begin{document}

 \section{Wstęp i cel pracy}
 
 \subsection{Wstęp}
 Potrzeba pomiaru czasu istnieje od początku istnienia ludzkiej cywilizacji. 
 Ludzie od zawsze starali się mierzyć czas, aby zorganizować swoje życie w sposób bardziej efektywny.
 
 Najprostsze i najstarsze metody pomiaru czasu to obserwacja zjawisk astronomicznych.
 Metody te, czyli np. obserwacja ruchu słońca, księżyca czy gwiazd, były stosowane przez tysiące lat,
  jednak była to metoda mało precyzyjna, która pchała wynalazców do poszukiwania nowych metod pomiaru czasu.
 
 Pierwszym zegarem stworzonym przez człowieka był zegar słoneczny, który wskazywał czas, 
 bazując na cieniu rzucanym przez słońce na tarczę zegara. Jednak metoda była zależna od długości dnia, 
 pory roku i nie pozwalała na pomiar czasu nocą.
 Jako rozwiązanie zaczęto stosować urządzenia, w których upływający czas wyznaczał stały i ciągły przepływ 
 substancji ciekłej lub sypkiej.
 
 Wraz z postępem technologicznym zaczęto stosować zegary mechaniczne, oparte na mechanicznych oscylatorach,
  takich jak wahadło czy sprężyna. Jednak ich precyzja była ograniczona do skali sekundowej, 
  a do tego pojawił się problem stabilności oscylatorów, które były podatne na zmiany temperatury i wilgotności.
   Pod koniec XIX wieku stabilność zegarów wahadłowych (ok. 0,01–0,001 sekundy na dobę) zaczynała graniczyć z tą, 
   dla której wpływ zmian g związanych z kształtem Ziemi miał już znaczenie \cite{st:czas}.
    Było to impulsem do poszukiwania metod opartych o rozwiązania elektroniczne.
 
 Powstały zegary kwarcowe, które miały własności niezależne od temperatury, były stabilne mechanicznie i chemicznie, 
 jednak ich stabilność nie była wystarczająca. Zegar po miesiącu mógł się spóźniać o kilka sekund. 
 Z powodu szybkiego rozwoju potrzebny był bardziej precyzyjny zegar o większej stabilności.
 
 Rozwiązaniem okazał się zegar atomowy, który bazuje na zjawisku przejścia pomiędzy dwoma poziomami energetycznymi
  w atomie. Wraz z pojawieniem się komputerów pojawiła się potrzeba ich synchronizacji, co doprowadziło do powstania
   serwerów czasu, które pozwalają na dokładną synchronizację czasu na całym świecie. 
   Powstała również możliwość bezprzewodowej synchronizacji czasu. Mimo zaawansowania technologicznego cały czas 
   występuje potrzeba opracowywania nowych rozwiązań.
 
 \subsection{Cel pracy}
 Celem pracy jest zaprojektowanie i wykonanie budzika opartego o lampy Nixie, który będzie synchronizowany z serwerem
  czasu. Mechanizm ten będzie wspierany modułem zegara czasu rzeczywistego wbudowanym w mikrokontroler. 
  Czas pobrany z serwera będzie wyświetlany na lampach Nixie. Odtwarzanie alarmu będzie realizowane za pomocą 
  źródła dźwięku umieszczonego na urządzeniu. Powinna być też możliwość wyłączenia alarmu za pomocą przycisku.
 
 Urządzenie będzie wykorzystywać aplikację realizującą interfejs użytkownika, przy zachowaniu części ustawień 
 bezpośrednio na urządzeniu. Aplikacja interfejsu użytkownika będzie umożliwiała ustawienie godziny alarmu i będzie hostowana na zewnętrznym serwerze. 
 Projekt zakłada stałe połączenie z siecią Wi-Fi.
 
 \end{document}
