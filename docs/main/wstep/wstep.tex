% Created 2024-10-16 śro 21:35
% Intended LaTeX compiler: pdflatex
\documentclass[../main.tex]{subfiles}

% \usepackage[a4paper, margin=3cm]{geometry}
% \usepackage{amssymb} // not working

\usepackage[T1]{fontenc}
\usepackage[utf8]{inputenc}
\usepackage{graphicx}
\usepackage{longtable}
\usepackage{wrapfig}
\usepackage{rotating}
\usepackage[normalem]{ulem}
\usepackage{amsmath}
\usepackage{capt-of}
\usepackage{hyperref}
\usepackage{siunitx}
\usepackage{float}
\usepackage[polish]{babel}

\graphicspath{{../}}
\author{Wojciech Paderewski}
\date{\today}
\title{Koncepcja ukladu}
\hypersetup{
 pdfauthor={Wojciech Paderewski},
 pdftitle={Koncepcja ukladu},
 pdfkeywords={},
 pdfsubject={},
 pdflang={Polish}}

\begin{document}

\section{Wstęp i cel pracy}
\subsection{Wstęp}
???
\subsection{Cel pracy}
Celem pracy jest zaprojektowanie i wykonanie budzika opartego o lampy Nixie, który będzie synchronizowany z serwerem czasu. 
Mechanizm ten będzie wspierany modułem zegara czasu rzeczywistego wbudowanym w mikrokontroler. Czas pobrany z serwera będzie wyświetlany na lampach Nixie.
Odtwarzanie alarmu będzie realizowane za pomocą źródła dźwięku umieszczonego na urządzeniu. Powinna być też możliwość wyłączenia alarmu za pomocą przycisku.

Urządzenie będzie wykorzystywać aplikację realizującą interfejs użytkownika, przy zachowaniu części ustawień bezpośrednio
na urządzeniu. Aplikacja interfejsu użytkownika będzie umożliwiała ustawienie godziny alarmu i będzie on hostowana na zewnętrznym serwerze.
Projekt zakłada stałe połączenie z siecią Wi-Fi. 

\end{document}
