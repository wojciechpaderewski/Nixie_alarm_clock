% Created 2024-10-16 śro 21:35
% Intended LaTeX compiler: pdflatex
\documentclass[../main.tex]{subfiles}

% \usepackage[a4paper, margin=3cm]{geometry}
% \usepackage{amssymb} // not working

\usepackage[T1]{fontenc}
\usepackage[utf8]{inputenc}
\usepackage{graphicx}
\usepackage{longtable}
\usepackage{wrapfig}
\usepackage{rotating}
\usepackage[normalem]{ulem}
\usepackage{amsmath}
\usepackage{capt-of}
\usepackage{hyperref}
\usepackage{siunitx}
\usepackage{float}
\usepackage[polish]{babel}

\graphicspath{{../}}
\author{Wojciech Paderewski}
\date{\today}
\title{Koncepcja ukladu}
\hypersetup{
 pdfauthor={Wojciech Paderewski},
 pdftitle={Koncepcja ukladu},
 pdfkeywords={},
 pdfsubject={},
 pdflang={Polish}}

\begin{document}

\section{Wstęp i cel pracy}
\subsection{Wstęp}
Głównym celem jaki został postawiony jest zrealizowanie budzika który synchronizuje się z wykorzystaniem zewnętrznego serwera czasu. ctd.
\subsection{Cel pracy}
Celem pracy jest zaprojektowanie i wykonanie budzika opartego o lampy Nixie, który będzie synchronizowany z serwerem czasu. 
Mechanizm ten będzie wspierany modułem RTC wbudowanym w mikrokontroler.

Projekt zakłada stałe połączenie z siecią Wi-Fi, ale istnieć będzie możliwość
działania w trybie offline, z wykorzystaniem modułu RTC wbudowanego w mikrokontroler, 
chociaż w przypadku tego trybu dokładność będzie mniejsza i nie będzie możliwe ustawienie alarmu. 

Urządzenie będzie wykorzystywać aplikację jako interfejs użytkownika, przy zachowaniu części ustawień bezpośrednio
na urządzeniu.
\end{document}
