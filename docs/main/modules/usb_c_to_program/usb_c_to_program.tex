\documentclass{article}
\usepackage[a4paper, margin=2.5cm]{geometry}
\usepackage{polski, graphicx, float, siunitx}
\setcounter{secnumdepth}{0}

\begin{document}
\subsection{Złącze USB-C do programowania}
\subsubsection{Dlaczego USB-C?}
Złącze zostało wybrane tylko jako narzędzie do programowania, nie będzie ono zasilać zegara, ponieważ przez zastosowanie paska LED, zegar będzie pobierał więcej prądu niż jest w stanie dostarczyć złącze USB-C, 
bez wykorzystania power delivery.
\\\\
Nie chcemy korzystać z power delivery, ponieważ nie jest on kompatybilny ze wszystkimi zasilaczami USB-C. Przez co zegar nie mogłby być zasilany np z złącza usb w komputerze,
a wypadku takiego podłączenia, zegar mógłby przestać działać, dlatego zasilanie zegara będzie zrealizowane przez złącze DC-plug 12V. 
\\\\
Standard usb-c jest obecnie najbardziej popularnym złączem w urządzeniach mobilnych,
więc jest to najwygodniejsze złącze do programowania. Złącze USB-C jest również rozwiązanie przyszłościowe.

\subsubsection{Opis podłączenia}
%TODO: napisać dlaczego mogę użyć lini bezpośrednio z usb-c podpiętych do esp32-s3
Złącze USB-C będzie podłączone bezpośrednio do ESP32-S3, ponieważ ESP32-S3 posiada wbudowany kontroler USB-C, 
który obsługuje protokół USB 2.0. ESP32-S3 posiada również wbudowany programator, więc nie potrzebujemy dodatkowego programatora.
\\\\
By ustawić napięcie komunikacji USB-C na 3.3V, zastosowano rezystory podciągające R1 i R2 o wartości 5.1k$\Omega$.
Do podłączenia wykorzystano parę różnicową by połączyć linie D+ i D- z ESP32-S3, w celu zminimalizowania zakłóceń CMN (Common Mode Noise).

\subsubsection{Zabezpieczenia ESD}

W celu zabezpieczenia linii przed przepięciami, zastosowano diody TVS PUSB3AB4Z firmy Nexperia. Diody te mają wystarczająco duże opakowanie by dało się je zlutować ręcznie, napięciem roboczym jest 3.3V,
a napięcie stabilizacji wynosi 5V. 
\\\\
Mimo że jest to napięcie wyższe niż napięcie zasilania ESP32-S3, to nie powinno to stanowić problemu, 
ponieważ napięcie to pojawi się na krótki czas, a sam esp32-s3 ma również wbudowane zabezpieczenia przed przepięciami.
\\\\
Wewnętrzne zabezpieczenia według noty katalogowej ESP32-S3:

\begin{itemize}
    \item Test Standard JS-001; HBM (Human Body Mode) ± 2000 V
    \item Test Standard JS-002; CDM (Charged Device Model) ± 1000 V
\end{itemize}

Wynika z tego, że złącze USB-C w dość dobry sposób jest zabezpieczone przed przepięciami.

\subsubsection{Schemat}
\begin{figure}[H]
    \centering
    \includegraphics[width=0.8\textwidth]{usb-c_schemat.png}
    \caption{Schemat złącza USB-C do programowania}
\end{figure}

\end{document}
