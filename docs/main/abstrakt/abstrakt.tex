% Created 2024-10-16 śro 21:35
% Intended LaTeX compiler: pdflatex
\documentclass[../main.tex]{subfiles}

% \usepackage[a4paper, margin=3cm]{geometry}
% \usepackage{amssymb} // not working

\usepackage[T1]{fontenc}
\usepackage[utf8]{inputenc}
\usepackage{graphicx}
\usepackage{longtable}
\usepackage{wrapfig}
\usepackage{rotating}
\usepackage[normalem]{ulem}
\usepackage{amsmath}
\usepackage{capt-of}
\usepackage{hyperref}
\usepackage{siunitx}
\usepackage{float}
\usepackage[polish]{babel}

\graphicspath{{../}}
\author{Wojciech Paderewski}
\date{\today}
\title{Koncepcja ukladu}
\hypersetup{
 pdfauthor={Wojciech Paderewski},
 pdftitle={Koncepcja ukladu},
 pdfkeywords={},
 pdfsubject={},
 pdflang={Polish}}

\begin{document}

\selectlanguage{polish}
\begin{abstract}
  Głównym celem, jaki został postawiony jest zrealizowanie budzika, który synchronizuje się z wykorzystaniem zewnętrznego serwera czasu. W 
  pierwszej części pracy przedstawiono niezbędne zagadnienia teoretyczne potrzebne do zrozumienia etapu projektowania układu. Następnie przedstawiono koncepcję układu oraz
  podzielono go na moduły, które zostały opisane szczegółowo. Na tej podstawie zaprojektowano i wykonano płytkę PCB, po czym zmontowano układ.
  W końcowej części pracy omówiono oprogramowanie stworzone do obsługi budzika. Na końcu znajduje się opis testów oraz wnioski z nich wynikające. Układ projektowy
  spełnia wszystkie założenia projektowe i działa zgodnie z oczekiwaniami.
\end{abstract}
\textbf{Słowa kluczowe:} Nixie, ESP32, Wi-Fi, Home Assistant, NTP \\
\textbf{Dziedzina nauki i techniki, zgodne z wymogami OECD:} nauki inżynieryjno-techniczne: automatyka, elektronika, elektrotechnika i technologie kosmiczne
\newpage
\selectlanguage{english}
\begin{abstract}
  The main goal is to create an alarm clock that synchronizes with an external time server. IN 
  The first part of the work presents the necessary theoretical issues needed to understand the system design stage. Then the concept of the system was presented
  it was divided into modules that were described in detail. On this basis, a PCB was designed and made, and the system was assembled.
  The final part of the work discusses the software created to operate the alarm clock. At the end there is a description of the tests and the conclusions drawn from them. Design layout
  meets all design assumptions and works as expected.
\end{abstract}
\textbf{Keywords:} Nixie, ESP32, Wi-Fi, Home Assistant, NTP
\selectlanguage{polish}
\end{document}
