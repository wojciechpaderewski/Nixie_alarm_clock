% Created 2024-10-16 śro 21:35
% Intended LaTeX compiler: pdflatex
\documentclass[../main.tex]{subfiles}

% \usepackage[a4paper, margin=3cm]{geometry}
% \usepackage{amssymb} // not working

\usepackage[T1]{fontenc}
\usepackage[utf8]{inputenc}
\usepackage{graphicx}
\usepackage{longtable}
\usepackage{wrapfig}
\usepackage{rotating}
\usepackage[normalem]{ulem}
\usepackage{amsmath}
\usepackage{capt-of}
\usepackage{hyperref}
\usepackage{siunitx}
\usepackage{float}
\usepackage[polish]{babel}

\graphicspath{{../}}
\author{Wojciech Paderewski}
\date{\today}
\title{Koncepcja ukladu}
\hypersetup{
 pdfauthor={Wojciech Paderewski},
 pdftitle={Koncepcja ukladu},
 pdfkeywords={},
 pdfsubject={},
 pdflang={Polish}}

\begin{document}

\selectlanguage{polish}
\begin{abstract}
  W pracy przedstawiono proces projektowania zegara synchronizowanego
  z internetowym serwerem czasu. Zegar ten posiada funkcję budzika,
  współpracuje z serwerem Home Assistant, zaś wizualizacja czasu dokonywana
  jest za pomocą lamp Nixie. Przedstawiono niezbędne zagadnienia teoretyczne
  potrzebne do zrozumienia etapu projektowania układu. Następnie
  przedstawiono koncepcję układu oraz podzielono go na moduły, które
  zostały opisane szczegółowo. Omówiono proces projektowania i wykonania
  obwodów drukowanych oraz montażu układu. Zaprezentowano oprogramowanie
  stworzone do obsługi zegara. Opisano proces uruchomiania i testowania układu,
  w którym potwierdzono spełnienie przez układ wszystkich wymagań
  projektowych. Sformułowano wnioski dotyczące kierunków dalszych prac.  
\end{abstract}
\textbf{Słowa kluczowe:} Nixie, ESP32, Wi-Fi, Home Assistant, NTP \\
\textbf{Dziedzina nauki i techniki, zgodne z wymogami OECD:} nauki inżynieryjno-techniczne: automatyka, elektronika, elektrotechnika i technologie kosmiczne
\newpage
\selectlanguage{english}
\begin{abstract}
  The paper presents the process of designing a synchronized clock
  with an internet time server. This clock has an alarm clock function,
  works with the Home Assistant server, and time visualization is performed
  is using Nixie tubes. Necessary theoretical issues are presented
  needed to understand the system design stage. Then
  the concept of the system was presented and it was divided into modules, which
  have been described in detail. The design and implementation process was discussed
  printed circuits and system assembly. The software was presented
  created to operate the clock. The process of starting and testing the system is described,
  which confirmed that the system meets all requirements
  design. Conclusions were formulated regarding directions for further work.
\end{abstract}
\textbf{Keywords:} Nixie, ESP32, Wi-Fi, Home Assistant, NTP
\selectlanguage{polish}
\end{document}
