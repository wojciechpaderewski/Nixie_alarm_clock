% Created 2024-10-16 śro 21:35
% Intended LaTeX compiler: pdflatex
\documentclass[../main.tex]{subfiles}

% \usepackage[a4paper, margin=3cm]{geometry}
% \usepackage{amssymb} // not working

\usepackage[T1]{fontenc}
\usepackage[utf8]{inputenc}
\usepackage{graphicx}
\usepackage{longtable}
\usepackage{wrapfig}
\usepackage{rotating}
\usepackage[normalem]{ulem}
\usepackage{amsmath}
\usepackage{capt-of}
\usepackage{hyperref}
\usepackage{siunitx}
\usepackage{float}
\usepackage[polish]{babel}

\graphicspath{{../}}
\author{Wojciech Paderewski}
\date{\today}
\title{Koncepcja ukladu}
\hypersetup{
 pdfauthor={Wojciech Paderewski},
 pdftitle={Koncepcja ukladu},
 pdfkeywords={},
 pdfsubject={},
 pdflang={Polish}}

\begin{document}

\selectlanguage{polish}
\begin{abstract}
  Głównym celem jaki został postawiony jest zrealizowanie budzika który synchronizuje się z wyko-
  rzystaniem zewnętrznego serwera czasu. ctd.
\end{abstract}
\textbf{Słowa kluczowe:} Nixie, ESP32, Wi-Fi, Home Assistant, NTP
\textbf{Dziedzina nauki i techniki, zgodne z wymogami OECD:} nauki inżynieryjno-techniczne: automatyka, elektronika, elektrotechnika i technologie kosmiczne
\newpage
\selectlanguage{english}
\begin{abstract}
  The main goal was to create an alarm clock that synchronizes with the use of
  using an external time server. ctd.
\end{abstract}
\textbf{Keywords:} Nixie, ESP32, Wi-Fi, Home Assistant, NTP
\selectlanguage{polish}
\end{document}
