% Created 2024-10-16 śro 21:35
% Intended LaTeX compiler: pdflatex
\documentclass[../main.tex]{subfiles}

% \usepackage[a4paper, margin=3cm]{geometry}
% \usepackage{amssymb} // not working

\usepackage[T1]{fontenc}
\usepackage[utf8]{inputenc}
\usepackage{graphicx}
\usepackage{longtable}
\usepackage{wrapfig}
\usepackage{rotating}
\usepackage[normalem]{ulem}
\usepackage{amsmath}
\usepackage{capt-of}
\usepackage{hyperref}
\usepackage{siunitx}
\usepackage{float}
\usepackage[polish]{babel}

\graphicspath{{../}}
\author{Wojciech Paderewski}
\date{\today}
\title{Obliczenia mocy}
\hypersetup{
 pdfauthor={Wojciech Paderewski},
 pdftitle={Obliczenia mocy},
 pdfkeywords={},
 pdfsubject={},
 pdflang={Polish}}

\begin{document}

By móc zaprojektować odpowiednie zasilanie dla całego układu, należy obliczyć szacunkową moc potrzebną do zasilania wszystkich komponentów.
Poza lampami Nixie, najbardziej obciążającym elementem będzie pasek LED oraz mikrokontroler, pozostałe elementy będą pobierały znikome ilości prądu.

Założono maksymalną długość paska LED na \SI{30}{\centi\meter}. Z deklaracji producenta paska LED wynika, że moc na metr wynosi \SI{18}{\watt}, co daje:
\begin{equation}
    P_{\text{LED}} = \SI{18}{\watt\per\meter} \cdot \SI{0.3}{\meter} = \SI{5.4}{\watt}
\end{equation}
Następnie obliczono prąd potrzebny do zasilenia paska LED przy napięciu \SI{5}{\volt}:
\begin{equation}
    I_{\text{LED}} = \frac{P_{\text{LED}}}{U_{\text{LED}}} = \frac{\SI{5.4}{\watt}}{\SI{5}{\volt}} = \SI{1.08}{\ampere}
\end{equation}

Następnie obliczono moc potrzebną do zasilania mikrokontrolera ESP32-S3. Według producenta maksymalny pobór prądu wynosi \SI{340}{\milli\ampere},
co przy napięciu zasilania \SI{3.3}{\volt} daje:

\begin{equation}
    P_{\text{ESP32}} = \SI{340}{\milli\ampere} \cdot \SI{3.3}{\volt} = \SI{1.122}{\watt}
\end{equation}

Następnie policzono prąd pobierany przez wszystkie lampy, których jest 6 sztuk, przy prądzie katodowym \SI{2}{\milli\ampere} każda, co daje:
\begin{equation}
    I_{\text{Nixie}} = 6 \cdot \SI{2}{\milli\ampere} = \SI{12}{\milli\ampere}
\end{equation}
Jednak należy dodać jeszcze prąd potrzebny do zasilania kropek oraz separatorów, oszacowano prąd kropki na \SI{1}{\milli\ampere} oraz separatora na \SI{1}{\milli\ampere}, co daje:
\begin{equation}
    I_{\text{Nixie}} = 6 \cdot \SI{2}{\milli\ampere} + 6 \cdot \SI{1}{\milli\ampere} + 2 \cdot \SI{1}{\milli\ampere} = \SI{20}{\milli\ampere}
\end{equation}
Następnie obliczono moc potrzebną do zasilania lampy nixie, przy napięciu \SI{220}{\volt} oraz prądzie \SI{20}{\milli\ampere}, zakładając 
sprawność przetwornicy na poziomie \SI{70}{\percent}:

\begin{equation}
    P_{\text{Nixie}} = \frac{U_{\text{Nixie}} \cdot I_{\text{Nixie}}}{\text{Sprawność}} = \frac{\SI{220}{\volt} \cdot \SI{20}{\milli\ampere}}{\SI{0.7}{}} = \SI{6.29}{\watt}
\end{equation}

Pozostałe komponenty będą pobierały znikome ilości prądu, więc nie będą brane pod uwagę w obliczeniach.
Szacunkowa moc potrzebna do zasilania całego układu wynosi:

\begin{equation}
    P_{\text{całkowita}} = P_{\text{LED}} + P_{\text{ESP32}} + P_{\text{Nixie}} = \SI{5.4}{\watt} + \SI{1.122}{\watt} + \SI{6.29}{\watt} = \SI{12.812}{\watt}
\end{equation}

Szacunkowa moc potrzebna do zasilania całego układu wynosi około \SI{13}{\watt}, 

\end{document}
