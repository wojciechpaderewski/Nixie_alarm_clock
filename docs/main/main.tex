\documentclass[twoside]{article}
\usepackage{graphicx, siunitx, float, polski, csquotes, pdfpages, chngcntr, pgfplots, amsmath}
\usepackage[a4paper, top=2.5cm, bottom=2.5cm, inner=3.5cm, outer=2.5cm]{geometry}
\usepackage[hyphens]{url}
\usepackage[pdfusetitle]{hyperref}
\usepackage[backend=biber, sorting=none]{biblatex}
\usepackage[polish, english]{babel}

\addbibresource{sources.bib}
% \appto{\bibsetup}{\raggedright}
% \appto{\bibsetup}{\sloppy}
\hypersetup{hidelinks}
\counterwithin{figure}{section}
\numberwithin{equation}{section}
\renewcommand{\thefigure}{\thesection.\arabic{figure}}
\renewcommand{\thetable}{\thesection.\arabic{table}}
\linespread{1.3}
% \cite{st:ism330dhcx}
\usepackage{subfiles}

\title{Budzik synchronizowany przez Wi-Fi}
\author{Wojciech Paderewski}

\begin{document}
\selectlanguage{polish}
\includepdf[pages=1]{strona_tytulowa/strona_tytulowa.pdf}
\thispagestyle{empty}
\cleardoublepage

\subfile{abstrakt/abstrakt}
\cleardoublepage

\tableofcontents
\newpage
\subfile{skroty/skroty}
\cleardoublepage

\subfile{wstep/wstep}
\newpage

\section{Lampy nixie}
\subfile{lampy_nixie/lampy_nixie}
\label{sec:nixie}
\newpage


\section{Mikrokontroler ESP32-S3}
\label{sec:esp32}
\subfile{esp32/esp32}
\newpage


\section{Serwery czasu}
\label{sec:serwery_czasu}
\subfile{serwery_czasu/serwery_czasu}
\newpage


\section{Koncepcja układu}
\subfile{koncepcja_ukladu/koncepcja_ukladu}
\newpage

\section{Realizacja}
\label{sec:realizacja}
W tym rozdziale, opisano szczegółową koncepcję układu, która została opracowana na podstawie analizy przeprowadzonej w rozdziałach \ref{sec:nixie} i \ref{sec:esp32} 
oraz po zapoznaniu się z ofertą sklepów elektronicznych.
Zostały określone konkretne rozwiązania projektowe, które będą wykorzystane oraz wykonano niezbędne obliczenia,
które pozwoliły na wybór odpowiednich komponentów na dalszym etapie projektowania.

\subsection{Szczegółowa koncepcja układu}
\subfile{realizacja/szczegolowa_koncepcja_ukladu/szczegolowa_koncepcja_ukladu}
\newpage

\subsection{Test lamp}
\subfile{realizacja/test_lamp/test_lamp}
\newpage

\subsection{Obliczenia mocy}
\label{sec:obliczenia_mocy}
\subfile{realizacja/obliczenia_mocy/obliczenia_mocy}
\newpage

\section{Realizacja modułów}
Po określeniu szczegółowej koncepcji układu, przystąpiono do realizacji poszczególnych modułów, które zostały opisane w rozdziale \ref{sec:realizacja}.
Każdy z modułów ma opisany dobór komponentów do realizacji, funkcjonalności modułu oraz sposób podłączenia.

\subsection{Sterowanie lampami Nixie}
\subfile{budowa_ukladu/nixie/nixie}
\newpage

\subsection{Przetwornica 12V na wysokie napięcie}
\subfile{budowa_ukladu/12V_to_HV_conv/12V_to_HV_conv}
\newpage

\subsection{Przetwornica 12V na 5V}
\subfile{budowa_ukladu/12V_to_5V_conv/12V_to_5V_conv}
\newpage

\subsection{Złącze zasilania}
\subfile{budowa_ukladu/DC_Plug/DC_Plug}
\newpage

\subsection{Złącze do programowania}
\label{sec:usb_c_to_program}
\subfile{budowa_ukladu/usb_c_to_program/usb_c_to_program}
\newpage

\subsection{Buzzer}
\subfile{budowa_ukladu/buzzer/buzzer}
\newpage

\subsection{Enkoder}
\subfile{budowa_ukladu/encoder/encoder}
\newpage

\subsection{LDO 5V na 3.3V}
\subfile{budowa_ukladu/LDO/ldo.tex}
\newpage

\subsection{Złącze do uruchamiania układu}
\subfile{budowa_ukladu/UART/uart}
\newpage

\subsection{Złącze do paska LED}
\subfile{budowa_ukladu/LED/led}
\newpage

\subsection{Mikrokontroler ESP32-S3}
\subfile{budowa_ukladu/ESP32/esp32}
\newpage

\section{Projekt i montaż płytki drukowanej}

\subfile{montaz/montaz}
\newpage

\section{Oprogramowanie}
\subfile{program/program}
\newpage

\section{Testowanie układu}
\label{sec:testy}
\subfile{testy/testy}
\newpage

\section{Podsumowanie}
\subfile{podsumowanie/podsumowanie}
\newpage

\printbibliography[]
\addcontentsline{toc}{section}{Bibliografia}
\newpage
\listoffigures
\addcontentsline{toc}{section}{Spis rysunków}
\newpage
\listoftables
\addcontentsline{toc}{section}{Spis tabel}
\end{document}
