% Created 2024-10-16 śro 21:35
% Intended LaTeX compiler: pdflatex
\documentclass[../main.tex]{subfiles}

% \usepackage[a4paper, margin=3cm]{geometry}
% \usepackage{amssymb} // not working

\usepackage[T1]{fontenc}
\usepackage[utf8]{inputenc}
\usepackage{graphicx}
\usepackage{longtable}
\usepackage{wrapfig}
\usepackage{rotating}
\usepackage[normalem]{ulem}
\usepackage{amsmath}
\usepackage{capt-of}
\usepackage{hyperref}
\usepackage{siunitx}
\usepackage{float}
\usepackage[polish]{babel}

\graphicspath{{../}}
\author{Wojciech Paderewski}
\date{\today}
\title{Koncepcja ukladu}
\hypersetup{
 pdfauthor={Wojciech Paderewski},
 pdftitle={Koncepcja ukladu},
 pdfkeywords={},
 pdfsubject={},
 pdflang={Polish}}

\begin{document}
Po przetestowaniu wymaganych modułów opisanych w rozdziale \ref{sec:testy} przystąpiono do pisania właściwego programu do obsługi budzika.
Program został napisany w języku C++ z wykorzystaniem platformy PlatformIO, wykorzystując Arduino framework.
\section{Struktura programu}
Struktura programu została podzielona na poszczególne moduły. Każdy z nich odpowiada za inną funkcjonalność budzika.
Struktura programu została przedstawiona na diagramie ??.

\end{document}
